\begin{frame}
  \frametitle{Getting started}
    \begin{itemize}[<+->]
      \item Prologue
      \item Grammar
      \item Epilogue
    \end{itemize}
\end{frame}

\begin{frame}
  \frametitle{Prologue}
    \begin{itemize}[<+->]
      \item Declarations
      \begin{itemize}[<+->]
        \item \%defines
        \item \%expect 0
        \item \%skeleton \"lalr1.cc\"
      \end{itemize}
      \item Includes
      \begin{itemize}[<+->]
        \item C code at the top of the parser
      \end{itemize}
      \item Token list
      \begin{itemize}[<+->]
        \item Token list
        \item #define
        \item Prototypes (yylex)
      \end{itemize}
  \end{itemize}
\end{frame}

\begin{frame}
  \frametitle{Grammar}
    \begin{itemize}[<+->]
      \item Rules
      \item Actions
    \end{itemize}
\end{frame}

\begin{frame}
  \frametitle{Grammar Rules}
    \begin{itemize}[<+->]
      \item Similar to BNF
      \item No compact way to describe optional elements/lists
      \item Beware of the \%empty!
    \end{itemize}
\end{frame}

\begin{frame}
  \frametitle{Grammar Actions}
    \begin{itemize}[<+->]
      \item C/C++ code run when the rule is recognized
      \item Create an AST, compute the result of an expression...
      \item Usually at the end
      \item Can be mid-rule actions (lexer context change, ...)
    \end{itemize}
\end{frame}


\begin{frame}
  \frametitle{Epilogue}
    \begin{itemize}[<+->]
      \item Raw C/C++ code included at the end of the parser
      \item main() to make it a stand-alone parser
    \end{itemize}
\end{frame}


\begin{frame}
  \frametitle{Manual}
    \begin{itemize}[<+->]
      \item GNU Bison manual
      \item Complete description of the parser
      \item Thorough examples (C/C++)
      \item Explanation of every option, parameter, ...
    \end{itemize}
\end{frame}


%FIXME: C++ specifics

